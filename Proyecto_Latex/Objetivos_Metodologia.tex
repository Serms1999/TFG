%!TEX root = TFG.tex

\lstset{frame=single,basicstyle=\ttfamily\small}

\chapter{Análisis de objetivos y metodología} \label{chap:meto}

En esta sección definiremos los objetivos del trabajo y se realizará un análisis sobre las decisiones tomadas.

\section{Objetivos}

El objetivo de este trabajo será entrenar un modelo de machine learning que sea capaz de identificar cada dispositivo respecto de los otros que se encuentran bajo análisis. Para realizar esto mediremos la desviación de los respectivos relojes (\textit{clock skew}).

Esta desviación se ver como la diferencia entre un reloj teóricamente exacto $c_{ex}$ y el reloj del dispositivo a analizar $c$ en un tiempo $t$. Con esto definimos la desviación del reloj $c$ en el tiempo $t$ como:
\begin{gather*}
    offset(c(t)) = c(t) - c_{ex}(t)
\end{gather*}
Una vez tengamos esta desviación de reloj para cada uno de los dispositivos, estudiaremos como varía esta desviación a lo largo del tiempo. Para esta labor definimos:
\begin{gather*}
    \Delta offset\left(c\left(t\right)\right) = offset\left(c\left(t\right)\right) - offset\left(c\left(t - 1\right)\right)
\end{gather*}

Una vez tenemos los datos recogidos, estudiaremos distintos valores estadísticos que, presumiblemente, nos aportarán información para distinguir los distintos dispositivos.

Por último analizaremos un conjunto de algoritmos de Machine Learning entrenados con un subconjunto estos datos, para evaluar su capacidad de identificar los dispositivos.
