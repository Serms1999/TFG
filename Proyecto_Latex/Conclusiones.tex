%!TEX root = TFG.tex

\chapter{Conclusiones y vías futuras} \label{chap:conclu}

En este proyecto hemos diseñado un modelo capaz de clasificar dispositivos con los que nos estamos comunicando en base a pequeñas diferencias en la fabricación de los componentes, que altera el tiempo que tardan en ejecutar una cierta tarea.

En la primera parte del proyecto hemos visto como para obtener la precisión en los tiempos que queríamos hemos tenido que usar el protocolo TCP y enviar las marcas de tiempo en el cuerpo del paquete. 

En este momento se vio que el interno es susceptible de ser alterado por procesos externos para que esté sincronizado en todo momento con el resto de los dispositivos (protocolo NTP), por este motivó este servicio tuvo que ser desactivado antes de realizar ninguna captura de paquetes, pues los diferencias entre tiempos no serían las propias del dispositivo.

Una vez desactivado el servicio, aún se obtenían datos que no eran correctos debido a que se estaba usando un reloj del sistema que podía ser modificado. Este reloj fue cambiado por un reloj que no fuera modificable (\texttt{steady\_clock}) y con eso los datos fueron más precisos.

Se realizaron capturas tanto en secuencial como en paralelo de la desviación de los relojes de los dispositivos, de las cuales se obtuvieron sus incrementos en cada momento. Con estos incrementos y una ventana deslizante de 1 minuto se obtienen variables estadísticas que servirán para entrenar los modelos.

Después de ver los resultados de los modelos con los conjuntos de entrenamiento/validación y analizando los posibles usos del sistema implementado se considera que es mejor quedarse con la muestra paralela. 

Por último entrenamos el modelo elegido, Random Forest, con los datos de la muestra paralela y los hiperparámetros que se consideraron mejores cuando se realizó el entrenamiento los conjuntos de entrenamiento/validación. De este modelo obtenemos unos resultados finales de 99.44\% en el valor de accuracy.

Como posibles vías futuras de este trabajo estaría el desarrollo de un modelo a tiempo real de este sistema. Para este cometido se debería tener una copia local de las huellas que generan ciertos dispositivos para poder compararlos con los que estamos recibiendo en ese momento y así comprobar si se trata de un atacante.

Para realizar este sistema a tiempo real también habría que crear mecanismos que permitan al sistema actualizarse con nuevos datos, y con ello generar nuevas huellas para los dispositivos. También habría que modificar los modelos de machine learning debido a que para que el sistema funcione a tiempo real, estos deberían actualizarse. 


