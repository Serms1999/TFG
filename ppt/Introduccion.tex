\section{Introducción}

%Normal Slide (copy, paste and modify this slide for longer presentations)
\begin{frame}
\frametitle{\secname} %Title
\framesubtitle{} %Subtitle
\rmfamily %Font
\color{black} %Color
\begin{itemize}
    \item Los dispositivos IoT traen consigo numerosos \rojoUMU{riesgos de seguridad}.
    \item Métodos tradicionales de identificación:
        \begin{itemize}
            \item Dirección IP.
            \item Dirección MAC.
            \item Certificados digitales.
        \end{itemize}
    \item Limitaciones:
        \begin{itemize}
            \item Direccionamiento dinámico.
            \item Modificaciones del usuario.
            \item Robo de credenciales.
        \end{itemize}
\end{itemize}
\end{frame}

%Normal Slide (copy, paste and modify this slide for longer presentations)
\begin{frame}
\frametitle{\secname} %Title
\framesubtitle{} %Subtitle
\rmfamily %Font
\color{black} %Color
\begin{itemize}
    \item Identificación \rojoUMU{individual} de los dispositivos.
        \begin{itemize}
            \item Mayor seguridad.
        \end{itemize}
    \item Diferencias en la \rojoUMU{fabricación} de los dispositivos.
        \begin{itemize}
            \item Diferentes tiempos en realizar un mismo proceso.
            \item Desviaciones en el reloj interno.
        \end{itemize}
    \item Objetivos de este trabajo:
    \begin{itemize}
        \item Diseñar una \rojoUMU{arquitectura} para este sistema. 
        \item Recolectar marcas de tiempo de los dispositivos.
        \item Generar \rojoUMU{huellas estadísticas} de los dispositivos.
        \item \rojoUMU{Machine Learning} para evaluar el proceso de identificación.
    \end{itemize}
\end{itemize}
\end{frame}