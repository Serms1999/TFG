\section{Estado del arte}

%Normal Slide (copy, paste and modify this slide for longer presentations)
\begin{frame}
\frametitle{\secname} %Title
\framesubtitle{} %Subtitle
\rmfamily %Font
\color{black} %Color
\vspace{0.5cm}
\begin{table}
    \centering
    \resizebox{\textwidth}{!}{
        \begin{tabular}{C{0.32\textwidth}C{0.25\textwidth}C{0.45\textwidth}C{0.34\textwidth}C{0.54\textwidth}}
            \toprule
            Referencia & Tipo de identificación & Enfoque & Tipo de aprendizaje & Resultados \\
            \midrule
            Pascal Oser et al. & Tipo de dispositivo & Machine Learning & Supervisado & 99.76\% de accuracy y 97.03\% de precisión \\
            \addlinespace\addlinespace
            Salma Hamad et al. & Individual & Machine Learning & Supervisado & 89\% de accuracy \\
            \addlinespace\addlinespace
            Ahmet Aksoy et al. & Tipo y modelo del dispositivo & Machine Learning & Supervisado & Entre 42.2\% y 100\% de accuracy, con un promedio de 82\% \\
            \addlinespace\addlinespace
            Hossein Jafari et al. & Individual & Machine Learning & Supervisado & 96.3\% de accuracy en DNN, 94.7\% de accuracy en CNN y 76\% de accuracy en LSTM \\
            \addlinespace\addlinespace
            Fabian Lanze et al. & Modelo del dispositivo & Análisis estadístico (regresión lineal) & - & Método no válido para identificar unívocamente un dispositivo. \\
            \addlinespace\addlinespace
            Yair Meidan et al. & Tipo y modelo & Machine Learning & Supervisado & 99.28\% de accuracy \\
            \addlinespace\addlinespace
            Loh Chin Choong Desmond et al. & Individual & Machine Learning & No supervisado & Entre un 70\% y 80\% de accuracy\\
            \addlinespace\addlinespace
            Este trabajo & Individual & Machine Learning & Supervisado y No supervisado & 99.38\% de Accuracy, 99.39\% de Recall y 99.38\% de $f$-score\\
            \bottomrule
        \end{tabular}
    }
    \caption{Resultados en el estado del arte}
\end{table}
\end{frame}

%Normal Slide (copy, paste and modify this slide for longer presentations)
\begin{frame}
\frametitle{\secname} %Title
\framesubtitle{} %Subtitle
\rmfamily %Font
\color{black} %Color
\begin{figure}
    \centering
    \subfloat[Propuesta de Ahmet Aksoy et al. para identificar \rojoUMU{modelos}.]{\resizebox{0.5\textwidth}{!}{\begin{tikzpicture}
    \node[draw, fill = azulclasificador, minimum width = 4.5cm] (0) {Hue Classifier};
    \node[draw, fill = azulclasificador, minimum width = 4.5cm] (1) [left = of 0] {Device Genre Classifier};
    \node[draw, fill = azulclasificador, minimum width = 4.5cm] (2) [above = 0.4cm of 0] {EdimaxPlug Classifier};
    \node[draw, fill = azulclasificador, minimum width = 4.5cm] (3) [above = 0.4cm of 2] {D-Link Classifier};
    \node[draw, fill = azulclasificador, minimum width = 4.5cm] (4) [below = 0.4cm of 0] {TP-Link Classifier};
    \node[draw, fill = azulclasificador, minimum width = 4.5cm] (5) [below = 0.4cm of 4] {Smarter Classifier};
    \node[draw, diamond, shape aspect = 2, fill = rojoclasificador] (6) [below = 0.4cm of 5] {Result};
    \node[draw, diamond, shape aspect = 2, fill = rojoclasificador] (7) [right = of 0] {Result};
    \node[draw, tape, tape bend top = none, tape bend height = 6pt, text width = 2.3cm, align = center, fill = verdeclasificador] (8) [above = 1.72cm of 1.west, anchor = west] {Arff file \\ (All devices)};
    
    \draw (1) -- (0);
    \draw (1.north) to [bend left = 18] (2.west);
    \draw (1.north) to [bend left] (3.west);
    \draw (1.south) to [bend right = 18] (4.west);
    \draw (1.south) to [bend right] (5.west);
    \draw (1.south) to [bend right] (6.west);
    
    \draw (0) -- (7);
    \draw (3.east) to [bend left] (7.north); 
    \draw (2.east) to [bend left] (7.north west); 
    \draw (4.east) to [bend right] (7.south west); 
    \draw (5.east) to [bend right] (7.south); 
    
    \draw (8) -- (8 |- 1.north);
\end{tikzpicture}}}
    \subfloat[Propuesta de Salma Hamad et al. para identificar \rojoUMU{dispositivos}.]{\resizebox{0.5\textwidth}{!}{\begin{tikzpicture}
    \node[draw, text width = 2cm, align = center, minimum width = 2cm, fill = blancodiagrama, execute at begin node=\setlength{\baselineskip}{1em}] (1) {\footnotesize Sequence Capture};
    \node (2) [left = 3cm of 1] {};
    \node[draw, rounded corners, fill = azuldiagrama] (3) [below = of 1] {\footnotesize Finger Print};
    \node[draw, diamond, aspect = 2, fill = azuldiagrama] (4) [below = of 3] {\footnotesize Classifier};
    \node[draw, cylinder, rotate = 90, fill = verdediagrama, minimum height = 0.5cm, minimum width = 0.5cm, scale = 0.5] (5) at (-0.9,-3.85) {};
    \node[draw, rounded corners, fill = azuldiagrama] (6) [below = of 4] {\footnotesize Zoning};
    \node (7) [left = 3cm of 6] {};
    \node (8) [below = of 6] {};
    \node[draw, fill = verdediagrama] (9) [left = of 8] {\footnotesize Trusted};
    \node[draw, fill = naranjadiagrama] (10) [right = of 8] {\footnotesize Restricted};
    \node[draw, text width = 2cm, align = center, minimum width = 2cm, fill = blancodiagrama, execute at begin node=\setlength{\baselineskip}{1em}] (11) [below = of 8] {\footnotesize Random Capture};
    \node[draw, rounded corners, fill = azuldiagrama] (12) [below = of 11] {\footnotesize Finger Print};
    \node[draw, diamond, aspect = 2, fill = azuldiagrama] (13) [below = of 12] {\footnotesize Classifier};
    \node[draw, cylinder, rotate = 90, fill = verdediagrama, minimum height = 0.5cm, minimum width = 0.5cm, scale = 0.5] (14) at (-0.9,-12.73) {};
    \node[draw, ellipse, minimum height = 1.5cm, fill = verdediagrama] (15) [left = of 13] {\footnotesize Match};
    \node[draw, fill = rojodiagrama] (16) [left = 2.5cm of 9] {\footnotesize Rejected};
    \node[draw, fill = rojodiagrama] (17) [right = 2.5cm of 10] {\footnotesize Quarantine};
    \node (18) [above right = 4.7cm of 13] {};
    \node (19) [below right = 0.7cm of 13] {};
    
    \draw (2) -- (1) node [midway, above, text width = 2.5cm, align = center] {\footnotesize Unseen Device Network Traces};
    \draw (2) -- (1) node [midway, below, text width = 2.5cm, align = center] {\footnotesize 010111000011001};
    \draw (1) -- (3);
    \draw (3) -- (4);
    \draw (4.west) -| (16.north) node [pos=0.25, above] {\footnotesize Unknown Device};
    \draw (4) -- (6);
    \draw (7) -- (6) node [above, midway] {\scriptsize Authorisation Matrix};
    \draw (6) -- (9);
    \draw (6) -- (10);
    \draw (9) -- (11);
    \draw (10) -- (11);
    \draw (11) -- (12);
    \draw (12) -- (13);
    \draw (15) -- (13);
    \draw (13) -| (18.center) node [pos=0.25, above] {\footnotesize Mis-Classified} |- (6);
    \draw (13) |- (19.center) -| (17) node [pos=0.17, above] {\footnotesize Suspicious/Mis-Behaving};
\end{tikzpicture}}}
\end{figure}
\end{frame}

